\documentclass[aspectratio=169]{beamer}
\usetheme{Madrid}
\usepackage{physics}
\usepackage{bm}
\usepackage{pdfpages}
% Remove the automatic "Figure" label in captions for the presentation
\usepackage{caption}
\captionsetup[figure]{labelformat=empty}
\usepackage{xcolor} % ensure color package available
\definecolor{ff00ff}{HTML}{FF00FF} % define color #ff00ff
% Set background to black and all text to white
\definecolor{highlight}{HTML}{ffdb58} % new highlight color #ffdb58
\usecolortheme{default}
% set alerted text color (no bold)
%\setbeamercolor{alerted text}{bg=highlight}
% remove any bold setting: do NOT use \setbeamerfont{alerted text}{series=\bfseries}
% inline highlight: black text on colored background (with small padding)
\newcommand{\hlbg}[1]{\begingroup\setlength{\fboxsep}{3pt}\colorbox{highlight}{\strut\textbf{\textcolor{black}{#1}}}\endgroup}
% convenience macro for inline highlighted text (no bold)
\newcommand{\hl}[1]{{\color{highlight}#1}}

\definecolor{mywhite}{RGB}{255,255,255}
%\setbeamercolor{background canvas}{bg=white}
%\setbeamercolor{normal text}{fg=black}
% Define the blue color for footline
\definecolor{footlineblue}{HTML}{003DA5}

\setbeamercolor{frametitle}{bg=white, fg=footlineblue}
%\setbeamercolor{title}{bg=black, fg=mywhite}
%\setbeamercolor{section in toc}{fg=black}
%\setbeamercolor{subsection in toc}{fg=black}
%\setbeamercolor{item}{fg=black}
%\setbeamercolor{block title}{bg=black, fg=mywhite}
%\setbeamercolor{block body}{bg=black, fg=mywhite}

% Remove navigation symbols
\setbeamertemplate{navigation symbols}{}

% Remove the automatic section/subsection TOC at the top of each slide
\setbeamertemplate{headline}{}


% Show current slide number and total number of slides in footline

\definecolor{lightblue}{HTML}{D6E4F0}
% Show current slide number and total number of slides in footline
\begin{document}

\title{Chemfarm for collective quantum optics \\ Tutorial 2: Master equation and Quantum Jumps} 
%\email{nikita.leppanen@weizmann.ac.il}
\author{Nikita Leppenen}
\date{February 2026}

\begin{frame}
    \titlepage
\end{frame}

\begin{frame}
    \frametitle{Outline for today}
    \begin{itemize}
        \item Physics: Master equation and Quantum Trajectory methods
        
        \href{https://youtu.be/F4GHy1NcBFY?si=T-jDmrd-uuwJMGHL}{Lecture YouTube} Continuos measurement and Quantum Trajectory 
        \item Implementation
        \item Parallelization using Chemfarm
    \end{itemize}
\end{frame}

\includepdf[pages=1-8]{../materials/collective_atom_photon_slides_8-15.pdf}


\begin{frame}
    \frametitle{Zoo of Quantum Master equations for superradiance}
    \begin{subequations}
    \begin{equation}
        \pdv{\rho}{t} = -i(H_{\rm eff}\rho -\rho H_{\rm eff}^\dagger) + \sum_{n,m = 1}^N \gamma_{nm} \sigma_n \rho \sigma_m^\dagger
    \end{equation}
    \begin{equation}
        H_{\rm eff} = \sum_n(\Omega_n^* \sigma_n + \Omega_n \sigma_n^\dagger) +\sum_{n,m = 1}^N \qty(\Delta_{nm} - i\frac{\gamma_{nm}}{2}) \sigma_n^\dagger \sigma_m
    \end{equation}
    \end{subequations}
    \pause 
    Cases:
    \begin{itemize}
        \item Driven Dicke model: $\Omega_n = \Omega$, $\Delta_{nm} = \Delta$, $\gamma_{nm} = \gamma$
        \item Waveguide QED: $\Omega_n = \Omega e^{ikx_n}$, $\Delta_{nm} = \frac{\gamma_{\rm 1D}}{2}\sin(k|z_n-z_m|)$, $\gamma_{nm} = \gamma_{\rm 1D}\cos(k|z_n-z_m|)$
        \item Waveguide QED with individual decay: $\Omega_n = \Omega e^{ikx_n}$, $\Delta_{nm} = \frac{\gamma_{\rm 1D}}{2}\sin(k|z_n-z_m|)$, $\gamma_{nm} = \gamma_{\rm 1D}\cos(k|z_n-z_m|) + \gamma_{\rm s}\delta_{nm}$
        \item Cavity $z_n = n\lambda$
        \item Free space; 2D array; 3D array, etc. 
    \end{itemize}
\end{frame}

\begin{frame}
    \frametitle{Methods for solving the master equation}
    \begin{equation}
        \dot{\rho} = \mathcal{L} \rho, \qquad \mathcal{L} \in \text{Mat}(d^2, d^2), \qquad \text{Non-Hermitian symmetric matrix}
    \end{equation}
    \vspace{0.2 cm}
    \begin{columns}
        \column{0.33\textwidth}
        Differential equation solvers: Runge--Kutta, adaptive step size, etc. (Qutip \texttt{mesolve}, SciPy \texttt{solve\_ivp})
        
        \hlbg{Boring but straightforward}

        For steady state: solve $\mathcal{L}\rho_{\rm ss} = 0$ (Qutip \texttt{steadystate}, SciPy \texttt{null\_space})

        \hlbg{Boring but accessible}
        
        \pause
        \column{0.33\textwidth}
        Diagonalization: $\rho(t) = e^{\mathcal{L}t}\rho(0)$, eigenvalues of $\mathcal{L}$ give decay rates and oscillation frequencies (Qutip \texttt{eigenstates}, SciPy \texttt{eig})

        \hlbg{Today or Next time}

        Only a few eigenvalues are relevant for long-time dynamics, so we can use sparse diagonalization (SciPy \texttt{eigs}) 

        \hlbg{Very intresing: next time}

        \pause 
        \column{0.33\textwidth}
        Quantum Trajectories: \\ Monte Carlo wavefunction method, unraveling the master equation into stochastic trajectories of pure states (Qutip \texttt{mcsolve})

        \hlbg{Today}

        MPS + MPO methods (Tensor)

        \hlbg{Very advanced...}
    \end{columns}
    
\end{frame}

\begin{frame}
    \frametitle{Runge--Kutta time stepping}
    Goal: solve $\dot{y} = f(t,y)$ by stepping forward in time.
    \begin{equation}
        k_1 = f(t,y), \quad k_2 = f\!\left(t+\tfrac{h}{2},\, y+\tfrac{h}{2}k_1\right), \quad k_3 = f\!\left(t+\tfrac{h}{2},\, y+\tfrac{h}{2}k_2\right), \quad k_4 = f(t+h,\, y+h k_3)
    \end{equation}
    \begin{equation}
        y(t+h) = y(t) + \tfrac{h}{6}(k_1 + 2k_2 + 2k_3 + k_4)
    \end{equation}
    \begin{itemize}
        \item RK4 is 4th order: local error $\sim h^5$, global error $\sim h^4$
        \item Adaptive RK uses error estimates to adjust $h$ automatically
        \item In master equations, set $y = \mathrm{vec}(\rho)$ and solve $\dot{y} = \mathcal{L} y$
    \end{itemize}
    \centerline{\scriptsize Python direct implementation: \href{https://qutip.org/docs/4.2/guide/dynamics/dynamics-master.html}{Qutip \texttt{mesolve}}; SciPy \texttt{solve\_ivp} with method='RK45' or 'RK23'}
\end{frame}


\begin{frame}
    \frametitle{Driven Dicke model}
        $\Omega_n = \Omega$, $\Delta_{nm} = \Delta$, $\gamma_{nm} = \gamma$

    Master equation:
    \begin{equation}
        \pdv{\rho}{t} = -i\Omega\sum_n [\sigma_n + \sigma_n^\dagger, \rho] - i\Delta\sum_{n,m} [\sigma_n^\dagger \sigma_m, \rho] + \gamma\sum_{n,m} \qty(\sigma_n \rho \sigma_m^\dagger - \frac{1}{2}\{\sigma_n^\dagger \sigma_m, \rho\})
    \end{equation}
    \pause
    Collective spin operators: $S^\alpha = \sum_n \sigma_n^\alpha$; $S^+ = \sum_n \sigma_n^+$; $S^- = \sum_n \sigma_n^-$
    \begin{equation}
        \pdv{\rho}{t} = -i\Omega [S^x, \rho] - i\Delta [S^+ S^-, \rho] + \gamma\qty(S^- \rho S^+ - \frac{1}{2}\{S^+ S^-, \rho\})
    \end{equation}
    \pause 
    Non-Hermitian Hamiltonian: $H_{\rm eff} = \Omega S^x + \Delta S^+ S^- - i\frac{\gamma}{2} S^+ S^-$

    Master equation: $$\pdv{\rho}{t} = -i(H_{\rm eff}\rho - \rho H_{\rm eff}^\dagger) + \gamma S^- \rho S^+$$
\end{frame}
\begin{frame}
    \frametitle{Revealing the master equation with Quantum Trajectories}
    \begin{equation}
        \rho(t+\dd t) = \rho(t) - i(H_{\rm eff}\rho -\rho H_{\rm eff}^\dagger)\dd t + \gamma S^- \rho S^+ \dd t
    \end{equation}
    Krauss operators: $K_0 = 1 - iH_{\rm eff}\dd t$, $K_1 = \sqrt{\gamma \dd t} S^-$
    \begin{equation}
        \rho(t+\dd t) = K_0 \rho K_0^\dagger + K_1 \rho K_1^\dagger
    \end{equation}
    \pause
    Assume the system is in a pure state $\ket{\psi}$ at time $t$. Then we can write the state at time $t+\dd t$ as:
    \begin{equation}
        \ket{\psi(t+\dd t)} = \begin{cases}
            \frac{K_0 \ket{\psi}}{\norm{K_0 \ket{\psi}}} & \text{with probability } p_0 = \norm{K_0 \ket{\psi}}^2 \\
            \frac{K_1 \ket{\psi}}{\norm{K_1 \ket{\psi}}} & \text{with probability } p_1 = \norm{K_1 \ket{\psi}}^2  
        \end{cases}
    \end{equation}
    \pause
    Dynamics: $\rho(t+\dd t) = p_0 \ket{\psi_0}\bra{\psi_0} + p_1 \ket{\psi_1}\bra{\psi_1}$, where $\ket{\psi_0} = \frac{K_0 \ket{\psi}}{\norm{K_0 \ket{\psi}}}$ and $\ket{\psi_1} = \frac{K_1 \ket{\psi}}{\norm{K_1 \ket{\psi}}}$
\end{frame}

\begin{frame}
    \frametitle{Algorithm for Quantum Trajectories: Monte Carlo wavefunction method}
    {\scriptsize Realization in Python: \href{https://qutip.org/docs/4.5/guide/dynamics/dynamics-monte.html}{Qutip \texttt{mcsolve}}}
    \begin{enumerate}
        \item Initialize the system in a pure state $\ket{\psi_0}$
        \item Evolve the state under the non-Hermitian Hamiltonian $H_{\rm eff}$ for a time step $\dd t$: $\ket{\psi'} = e^{-iH_{\rm eff}\dd t} \ket{\psi}$
        \item Calculate the norm of the evolved state: $p_0 = \norm{\ket{\psi'}}^2$
        \item Generate a random number $r$ uniformly distributed in the interval $[0,1]$
        \item If $r < p_0$, the system evolves to $\ket{\psi_0} = \frac{\ket{\psi'}}{\norm{\ket{\psi'}}}$ (no jump occurs)
        \item If $r \ge p_0$, a quantum jump occurs, and the system evolves to $\ket{\psi_1} = \frac{K_1 \ket{\psi}}{\norm{K_1 \ket{\psi}}}$ (jump occurs)
        \item Repeat steps 2-6 for the desired number of time steps. Average over many trajectories to obtain the density matrix $\rho(t) = \frac{1}{N_{\rm traj}} \sum_{i=1}^{N_{\rm traj}} \ket{\psi_i(t)}\bra{\psi_i(t)}$
    \end{enumerate}
\end{frame}

\begin{frame}
    \frametitle{Advantages of the method} 
    \begin{itemize}
        \item Computational efficiency: The method allows for efficient simulation of open quantum systems by reducing the computational complexity associated with the density matrix formalism. Instead of evolving a density matrix, which has a size that scales as the square of the Hilbert space dimension, we evolve pure states, which scale linearly with the Hilbert space dimension.
        \item Physical insight: The method provides a clear physical picture of the dynamics of open quantum systems. The quantum jumps correspond to discrete events that can be interpreted as measurements or interactions with the environment, allowing for a more intuitive understanding of the system's behavior.
        \item Flexibility: The method can be applied to a wide range of open quantum systems, including those with time-dependent Hamiltonians, multiple decay channels, and non-Markovian dynamics.
    \end{itemize}
\end{frame}

\begin{frame}
    \frametitle{Implementation}
    \begin{enumerate}
        \item JupyterHub on ChemFarm: Notebook with example on Dicke model (Qutip \texttt{mcsolve})
        \item Python script for cavity model + PBS submission script for running on ChemFarm
    \end{enumerate}
\end{frame}
\end{document}