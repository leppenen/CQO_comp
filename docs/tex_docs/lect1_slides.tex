\documentclass[aspectratio=169]{beamer}
\usetheme{Madrid}
\usepackage{physics}
\usepackage{bm}
\usepackage{pdfpages}
% Remove the automatic "Figure" label in captions for the presentation
\usepackage{caption}
\captionsetup[figure]{labelformat=empty}
\usepackage{xcolor} % ensure color package available
\definecolor{ff00ff}{HTML}{FF00FF} % define color #ff00ff
% Set background to black and all text to white
\definecolor{highlight}{HTML}{ffdb58} % new highlight color #ffdb58
\usecolortheme{default}
% set alerted text color (no bold)
%\setbeamercolor{alerted text}{bg=highlight}
% remove any bold setting: do NOT use \setbeamerfont{alerted text}{series=\bfseries}
% inline highlight: black text on colored background (with small padding)
\newcommand{\hlbg}[1]{\begingroup\setlength{\fboxsep}{3pt}\colorbox{highlight}{\strut\textbf{\textcolor{black}{#1}}}\endgroup}
% convenience macro for inline highlighted text (no bold)
\newcommand{\hl}[1]{{\color{highlight}#1}}

\definecolor{mywhite}{RGB}{255,255,255}
%\setbeamercolor{background canvas}{bg=white}
%\setbeamercolor{normal text}{fg=black}
% Define the blue color for footline
\definecolor{footlineblue}{HTML}{003DA5}

\setbeamercolor{frametitle}{bg=white, fg=footlineblue}
%\setbeamercolor{title}{bg=black, fg=mywhite}
%\setbeamercolor{section in toc}{fg=black}
%\setbeamercolor{subsection in toc}{fg=black}
%\setbeamercolor{item}{fg=black}
%\setbeamercolor{block title}{bg=black, fg=mywhite}
%\setbeamercolor{block body}{bg=black, fg=mywhite}

% Remove navigation symbols
\setbeamertemplate{navigation symbols}{}

% Remove the automatic section/subsection TOC at the top of each slide
\setbeamertemplate{headline}{}


% Show current slide number and total number of slides in footline

\definecolor{lightblue}{HTML}{D6E4F0}
% Show current slide number and total number of slides in footline
\begin{document}

\title{Chemfarm for collective quantum optics \\ Tutorial 1: Overview and first steps} 
%\email{nikita.leppanen@weizmann.ac.il}
\author{Nikita Leppenen}
\date{February 2026}

\begin{frame}
    \titlepage
\end{frame}

\begin{frame}{Computers in Weizmann Institute}
    \begin{columns}
        \column{0.5\textwidth}
        \begin{figure}
            \includegraphics[width=0.8\textwidth]{img/1920px-Weizac_(1954-1964)_Front.png}
            \caption{Weizmann Institute's first computer, \\ Weizac, built in 1954.}
        \end{figure}
        \pause
        \column{0.5\textwidth}
        \hlbg{Now}
        \begin{itemize}
        \item WEXAC HPC: 34,000 CPU cores, NVidia GPUs, 11000 TB of storage
        \item Faculty of Math and Computer Science HPC: 2000 CPU cores, NVidia GPUs (with NVidia A 100, 80 GB of memory)
        \item \textbf{ChemFarm} 31648 CPU cores, 169 TB RAM, 19 nodes with NVidia GPU
        \end{itemize}

        \vspace{1cm}

        Useful links: \href{https://www.weizmann.ac.il/wit/services/hardware-infrastructure/high-performance-computing-wexac/high-performance-computing-wexac}{WEXAC HPC}, \href{https://www.weizmann.ac.il/chemistry/chemfarm/home}{ChemFarm}\\

        Only from WIS Network: \href{https://cf-wiki.weizmann.ac.il}{ChemFarm Wiki}
    \end{columns}
\end{frame}

\begin{frame}{What do people use HPC for?}
    HPC (High Performance Computing) is used for computationally intensive tasks, such as:
    \begin{itemize}
        \item Simulating complex physical systems (e.g., quantum optics, molecular dynamics)
        \item Data analysis and machine learning (e.g., large datasets, training neural networks)
        \item Computational chemistry (e.g., quantum chemistry calculations, drug discovery)
    \end{itemize}

    \pause 

    What my friends do:
    \begin{columns}
    \column{0.2\textwidth}
    \begin{figure}
        \includegraphics[width=0.5\textwidth]{img/Olga.jpg}
        \caption{Olga: computational biology (monte carlo simulation of Ising model for decision making)}
    \end{figure}
    \column{0.4\textwidth}
    \begin{figure}
        \includegraphics[width=0.25\textwidth]{img/Rafail.jpg}
        \includegraphics[width=0.25\textwidth]{img/Narek.jpg}
        \caption{Rafail and Narek: machine learning for computer vision (Train large model on many pictures using GPU) }
    \end{figure}
    \column{0.2\textwidth}
    \begin{figure}
        \includegraphics[width=0.5\textwidth]{img/Diana.jpg}
        \caption{Diana: condensed matter physics (running DFT calculations for topological materials)}
    \end{figure}
    \column{0.2\textwidth}
    \begin{figure}
        \includegraphics[width=0.5\textwidth]{img/Dima.jpg}
        \caption{Dima: machine learning for particle physics (writing code to analyze data from Large Hadron Collider)}
    \end{figure}
    \end{columns}
\end{frame}

\begin{frame}{What about collective quantum optics?}
\begin{columns}
\column{0.33\textwidth}
\centerline{Solution of Lindblad equation}
\begin{equation*}
    \dot{\rho} = {\cal L} \rho
\end{equation*}
Size of $\rho$ grows exponentially with the number of atoms (for $N$ two-level atoms, $\rho$ is a $2^N \times 2^N$ matrix). For $N=20$, $\rho$ has $2^{40} \approx 10^{12}$ elements, which is too large to store in memory.

We need to solve to find steady states, dynamics, correlation functions, etc.

\pause
\column{0.33\textwidth}
\centerline{Mean-field equations}

\vspace{0.3cm}
Experiments $N\sim 10^4-10^6$ atoms, so we need to solve mean-field equations for $N$ atoms, which is a system of $3N$ coupled nonlinear ODEs. 

Even at mean-field level computations can be challenging + some new methods (e.g., cluster mean-field, cumulant expansion) require solving even larger systems of equations.

\pause 
\column{0.33\textwidth}
\centerline{Large matrix diagonalization}

\vspace{0.9cm}
Finding ground states and first excited states of many-body Hamiltonians, which is a large matrix diagonalization problem. $N=20$ two-level atoms $\Rightarrow$ $2^{20} = 10^6$ 
Liouvivillian diagonalization: $4^N \times 4^N$ matrix. For $N=10$ $\Rightarrow$ $2^{20} = 10^6$ $4^{10} = 10^6$

GPUs can be used..
\end{columns}

\centerline{\hlbg{I believe there are much more and we should discuss}}
\end{frame}

\begin{frame}
    \frametitle{Plan for today}
    \begin{itemize}
        \item Look through the ChemFarm wiki and documentation to understand the available resources and how to access them
        \item Show examples of HPC usage in Mean-Field with Green Function
        \item Outline future directions and opportunities for collaboration
    \end{itemize}
\end{frame}

\begin{frame}
    \frametitle{ChemFarm Wiki}
    \begin{itemize}
        \item \href{https://cf-wiki.weizmann.ac.il}{ChemFarm Wiki} (Weizmann Network) is the main source of information about ChemFarm, including documentation, tutorials, and user guides.
        \item It contains information on how to access ChemFarm, how to use the job scheduler, and how to run different types of jobs (e.g., CPU, GPU).
        \item It also has a section with available software ()
    \end{itemize}

    Connect to cluster: ssh terminal command: \texttt{ssh <username>@chemfarm.weizmann.ac.il} (need to be on WIS network or use VPN); VS Code Remote SSH extension; JupyterHub;
    
    We will try OpenOnDemand, JupyterHub, and VS Code Remote SSH today.
\end{frame}
\includepdf[pages=1-8]{../materials/collective_atom_photon_slides_8-15.pdf}


\begin{frame}
    \frametitle{Mean-Field in waveguide QED or atomic array/cloud}
    Denoting $s_n = \ev{\sigma_n^-}$ and $s_n^z = \ev{\sigma_n^z}$ and neglecting quantum correlations, we get the following system of equations for $N$ atoms:
    \begin{equation}
\begin{cases}
    \dot{s}_n = -\frac{\gamma}{2}s_n +s_n^z \qty[\frac{\gamma_{\rm 1D}}{2}\sum_{m\neq n}D_{nm}s_m +i\Omega e^{ikz_n}] \\
    \dot{s}_n^z = -\gamma(s_n^z+1) - 2\qty[s_n^* \qty(\frac{\gamma_{\rm 1D}}{2}\sum_{m\neq n}D_{nm}s_m +i\Omega e^{ikz_n}) + c.c.]
\end{cases}
\end{equation}
\textbf{Geometry information:} $D_{nm} = -i\frac{3\gamma}{2}\lambda G(\bm r_n - \bm r_m)$

\pause
\textbf{Problem statement:} solve the system depending on the geometry. Find $s_n(t)$ and $s_n^z(t)$ for $n=1,...,N$ depending on $\Omega$, initial conditions.

\pause
\textbf{Where HPC comes in:} big memory ($N\sim 10^3$), NumPy for Greens Function, parallelization for parameter scans.  

\pause 
\textbf{Solution step:} calculate Green function using vectorized code, solve ODEs using \texttt{scipy.integrate.solve\_ivp} (parallelization for parameter scans), run on chemfarm
\end{frame}

\begin{frame}
    \frametitle{Parallel solver function}
    \begin{figure}
        \includegraphics[width=0.8\textwidth]{img/parallelsolve.png}
        \caption{n\_workers - number of parallel processes to use}
    \end{figure}
\end{frame}
\begin{frame}
    \frametitle{Vectorized 1D Greens function}
    \begin{figure}
        \includegraphics[width=0.8\textwidth]{img/1d_GF.png}
    \end{figure}
\end{frame}

\begin{frame}
    \frametitle{3D lattice with numpy}
    \begin{figure}
        \includegraphics[width=0.8\textwidth]{img/3d_mesh.png}
    \end{figure}
\end{frame}

\begin{frame}
    \frametitle{3D Greens function with numpy}
    \begin{figure}
        \includegraphics[width=0.8\textwidth]{img/3d_GF.png}
    \end{figure}
\end{frame}

\begin{frame}
    \vspace{-8cm}
    \hspace{1.5cm}\includegraphics[width=0.8\textwidth]{img/3d_GF.png}
\end{frame}
 
\end{document}